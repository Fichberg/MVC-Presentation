\documentclass{beamer}
\usepackage[T1]{fontenc}
\usepackage[utf8]{inputenc}
\usepackage{lmodern}
\usepackage[brazil]{babel}
\usepackage[labelformat=empty]{caption}
\usepackage{graphicx}
\usetheme{Luebeck}
\title[Padrão Model-View-Controller]{Padrão Model-View-Controller}
\author{Ana Luísa Losnak. Luiz Armesto. Renan Fichberg.}
\date{Novembro 4, 2014}
\institute{Instituto de Matemática e Estatística da Universidade de São Paulo (IME-USP)}
\begin{document}

\begin{frame}
\titlepage
\end{frame}

\begin{frame}
\frametitle{Introdução}
\begin{itemize}
	\item O que é um padrão?
	\item O padrão MVC
	\item Alguns frameworks que usam MVC ou derivados.
	\item Alguns padrões similares ao MVC
\end{itemize}
\end{frame}

%TODO: o que é um padrão (não há necessidade de dar exemplos. O tema por si só já é um exemplo)
\begin{frame}
\frametitle{O que é um padrão?}
	Em Engenharia de Software, um padrão de design é uma solução geral que pode ser reutilizada para resolver um problema recorrente de um contexto específico de Design de Software.
\end{frame}

%TODO: O padrão MVC
\begin{frame}
\frametitle{O padrão MVC}
\begin{itemize}
	\item Breve histórico
	\item Por que usar?
	\item O padrão MVC
\end{itemize}
\end{frame}

\begin{frame}
\frametitle{O padrão MVC}
	`An easy way to understand MVC: the model is the data, the view is the window on the screen, and the controller is the glue between the two.' \\
	\hfill \textit{- Connelly Barnes.}
\end{frame}


\begin{frame}
\frametitle{O padrão MVC}
\framesubtitle{Breve histórico}
	O cientista da computação norueguês Trygve Reenskaug é o pai do padrão MVC, que fez sua primeira aparição ao público no Smalltalk-80. Por muito tempo, não houve
	muitas informações acerca do padrão virtualmente, até o lançamento do primeiro \textit{paper} significativo, entitulado \textit{`A Cookbook for Using the Model-View-Controller User Interface Paradigm in Smalltalk -80'}, de Glenn Krasner e Stephen Pope, publicado em Agosto/Setembro de 1988, no \textit{`Journal Of Object Oriented Programming'}.
\end{frame}

\begin{frame}
\frametitle{O padrão MVC}
\framesubtitle{Por que usar?}
\begin{itemize}
	\item Interfaces de usuário estão propensas a mudar freqüentemente e, portanto, isto não pode ser uma tarefa complicada
	\item Ganhar flexibilidade: reduzir o entrelaçamento da interface de usuário com o núcleo funcional.
	\item Diminuir custos e propensão a erros na construção do sistema
\end{itemize}
\end{frame}

\begin{frame}
\frametitle{O padrão MVC}
\framesubtitle{}
	O padrão de arquitetura divide uma aplicação em três componentes que estão interconectados: o model, as views e os controllers, sendo os últimos dois os componentes que compreendem a interface de usuário.
\begin{center}
	\includegraphics[scale=0.4]{MVC.jpg}
\end{center}
\end{frame}

\begin{frame}
\frametitle{O padrão MVC}
\framesubtitle{A tríade: os três componentes do MVC}
\begin{itemize}
	\item Model: encapsula os dados e a funcionalidade. É independente de representações específicas das saídas de dados e do comportamento da entrada de dados.
	\item View: apresenta as informações para o usuário. Os dados apresentados são obtidos do model. Podem haver diversas views para o model.
	\item Controller: cada view tem um controller associado à ela. Controllers recebem eventos de entrada, como click de mouse ou leitura de teclado. Tais eventos são então traduzidos em requisições para o model ou para a view. É pelo controller que o usuário interage com o sistema.
\end{itemize}
\end{frame}

\begin{frame}
\frametitle{Alguns frameworks que usam MVC ou derivados.}
\framesubtitle{}
	\textbf{ActionScript 3} - Cairngorm, PureMVC.\\
	\textbf{ASP} - ASP Xtreme Evolution, Toika, AJAXED.\\
	\textbf{Java} - Apache Struts, Tapestry, VRaptor, Spring MVC, JSF.\\
	\textbf{Perl} - Catalyst.\\
	\textbf{PHP} - CakePHP, CodeIgniter, LightVC, Symfony.\\
	\textbf{Python} - Django.\\
	\textbf{Ruby} - Rails.\\
	\textbf{Smalltalk} - Smalltalk-80.\\
\end{frame}

\begin{frame}
\frametitle{Alguns frameworks que usam MVC ou derivados.}
\framesubtitle{Rails não é MVC puro!}
	Apesar do que se diz por aí, o famoso framework \textit{Rails}, para \textit{Ruby}, não é MVC, ou pelo menos, não o clássico, 
	mas sim um modelo derivado deste, chamado \textbf{Model 2}.
\end{frame}

\begin{frame}
\frametitle{Alguns frameworks que usam MVC ou derivados.}
\framesubtitle{Rails não é MVC puro!}
	No MVC clássico, um model pode notificar as views a respeito das mudanças que aconteceram (a partir do padrão \textit{Observer}). No Rails
	nós não notificamos as views a partir do model, o controller simplesmente passa o dado do model para as views e se encarrega da geração do HTML que 
	é posteriormente enviado ao browser, como exibido no esquema abaixo:
	\begin{center}
		\includegraphics[scale=0.35]{RailsMVC.jpg}
	\end{center} 
\end{frame}

\begin{frame}
\frametitle{Alguns padrões similares ao MVC}
\framesubtitle{O Model 2: um MVC do mundo do Java}
	A seguir, um diagrama do \textit{Design Pattern} \textbf{Model 2}, usado no framework \textit{Rails}, mencionado no último slide.
	\begin{center}
		\includegraphics[scale=0.2]{Model2.jpg}
	\end{center}
	As requisições do navegador do cliente são passadas para o controller. Na seqüência, o controller faz o que precisa para obter o conteúdo que deve ser exibido.
	Em seguida, o conteúdo é colocado na requisição, freqüentemente como um JavaBean, e é decidido para qual view o conteúdo será repassado para então, finalmente, esta
	view renderizá-lo.
\end{frame}

\begin{frame}
\frametitle{Alguns padrões similares ao MVC}
\framesubtitle{O Hierarchical Model–View–Controller (HMVC)}
	O \textbf{HMVC} é uma variação do MVC, similar a um outro chamado \textit{Presentation–Abstraction–Control (PAC)}, que surgiu como uma tentativa de resposta aos problemas de escalabilidade.
	\begin{center}
		\includegraphics[scale=0.175]{HMVC.jpg}
	\end{center}
	Cada tríade funciona independentemente e uma tríade pode fazer requisição de acesso à outra a partir dos seus controllers.
\end{frame}

\begin{frame}
\frametitle{Alguns padrões similares ao MVC}
\framesubtitle{MVC versus HMVC: algumas diferenças}
	\begin{itemize}
	\item MVC é mais simples.
	\item HMVC é mais escalável.
\end{itemize}
\end{frame}

\begin{frame}
\frametitle{Alguns padrões similares ao MVC}
\framesubtitle{O Movel-View-Adapter (MVA)}
	Uma outra variação do MVC é o \textbf{Model-View-Adapter (MVA)}, mostrado a seguir:
	\begin{center}
		\includegraphics[scale=0.4]{MVA.jpg}
	\end{center}
	Na imagem, as linhas pontilhadas representam comunicação indireta, através do uso do padrão \textit{Observer}, já as demais linhas são relações diretas.
\end{frame}

\begin{frame}
\frametitle{Alguns padrões similares ao MVC}
\framesubtitle{MVC versus MVA: algumas diferenças}
\begin{itemize}
	\item Desacoplamento total do model e da view no MVA.
	\item Toda a dinâmica da aplicação é centralizada no adapter.
\end{itemize}
\end{frame}

\begin{frame}
\frametitle{Alguns padrões similares ao MVC}
\framesubtitle{O Movel-View-Presenter (MVP)}
	Nesta outra variação do MVC, nomeada \textbf{Model-View-Presenter (MVP)}, a View é a responsável por manipular os eventos da interface de usuário, função do controller no MVC tradicional.
	\begin{center}
		\includegraphics[scale=0.3]{MVP.jpg}
	\end{center}
	O MVP surgiu, dentre outros motivos, com o objetivo de tentar facilitar a realização de testes de unidade automatizados (diferente do MVC tradicional, o presenter
	só interage com a view por meio de uma interface). Aqui, o presenter age sobre o model e a view, atualizando e recuperando dados do model e passando-os já preparados para a view, que apenas deverá exibi-los da maneira esperada.
\end{frame}

\begin{frame}
\frametitle{Alguns padrões similares ao MVC}
\framesubtitle{MVC versus MVP: algumas diferenças}
\begin{itemize}
	\item View está menos amarrada ao model no MVP. Amarrá-los é tarefa do presenter.\\
	\item Mais facilidade para realizar testes de unidade no MVP.\\
	\item Views complexas podem ter múltiplos presenters.
	\item No MVC, a parte lógica deve sempre ficar completamente isolada da view.
	\item Controller é baseado em comportamentos e pode ser compartilhado através das views.
\end{itemize}
\end{frame}

\end{document}
